%***************************************PREAMBLE***************************************
\documentclass[a4paper,12pt]{article}

\usepackage[utf8]{inputenc}
\usepackage[margin=0.7in]{geometry}
\usepackage[T1]{fontenc}
\usepackage{graphicx}
\usepackage{float}
\usepackage{setspace}
\usepackage{appendix}

%***************************************DOCUMENT***************************************

\begin{document}
	\fontfamily{ptm}\selectfont
	%%%%%%%%%%%%%%%%%%%%%%%%%%%%%%%%%%%%%%%COVERSHEET%%%%%%%%%%%%%%%%%%%%%%%%%%%%%%%%%%%%%%%
	\begin{titlepage}
		\setlength{\voffset}{-0.8in}
		\noindent \makebox[\textwidth]{\includegraphics[width=1.2\textwidth]{images/Coversheet_Header.png}}
	
			\vspace{15mm}
			
			\begin{center}
				{\Huge \textbf{COMP0037 \\ \vspace{10mm} Report}}
			
				\vspace{8mm}
			
				\begin{spacing}{1.8}
					{\huge Exploration in Unknown Environments}
				\end{spacing}
		
			
				\vspace{12mm}
			
				{\LARGE \textbf{Group AS}}
				
				\vspace{10mm}
				
				\begin{tabular}{ll}
					\underline{\textbf{Student Name}}  & \hspace{4mm} \underline{\textbf{Student number}} \vspace{2mm} \\
					Arundathi Shaji Shanthini & \hspace{4mm} 16018351 \\ 
					Dmitry Leyko & \hspace{4mm}  16021440\\ 
					Tharmetharan Balendran & \hspace{4mm} 17011729\\ 
				\end{tabular}
				
				\vspace{13mm}
				
				\begin{tabular}{ll}
					\textbf{Department:} &  Department of Electronic and Electrical Engineering\\ \vspace{3mm}
					\textbf{Submission Date:} &  17\textsuperscript{th} of March 2020
				\end{tabular}
			\end{center}
	\end{titlepage}
	%%%%%%%%%%%%%%%%%%%%%%%%%%%%%%%%%%%%%%
	
	\pagebreak
	
	\tableofcontents
	
	\pagebreak
	
	%%%%%%%%%%% PART 1 %%%%%%%%%%%%%%%%%
	\section{Implement Reactive Planner }
		\begin{figure}[H]
			\label{ReactivePlannerFlowchart}
			\centering
			
			\includegraphics[scale=0.13]{images/ReactivePlannerFlowchart.png}
			\caption{Flowchart for the reactive planner}
		\end{figure}
	
		A reactive planner is able to adapt its path based on the information it obtains about the environment as it explores it. The high-level functionality of a general reactive planner is described in the flowchart in Fig.\ref{ReactivePlannerFlowchart}. The reactive planner initially utilizes the available occupancy grid and assumes any unknown cells to be free. The planner then plans a path using this assumption and starts to traverse the path. The robot will explore the environment as it traverses the path and if new information suggests that the currently planned path is no longer traversable, a new path is planned using the new occupancy grid. This process is repeated until the robot arrives at the goal.
		\\
		As a means of testing the code the robot was set to visit a list of goals on the factory map. The final mapper node occupancy grid is shown in Fig.\ref{mapperNodeOccupancyGrid}. It is clear to see that there are inaccuracies in the mapping of the world as well as some areas which have still not been explored completely. The inaccuracies occur as a result of rounding errors and other noise due to timing mismatch (caused by networking delays), these errors are mostly corrected as the robot re-explores an area. 

		\begin{figure}[H]
			\label{mapperNodeOccupancyGrid}
			\centering
			
			\includegraphics[scale=0.3]{images/mapperNodeOccupancyGrid.png}
			\caption{Flowchart for the reactive planner}
		\end{figure}

	%%%%%%%%%%%%%%%%%%%%%%%%%%%%%%%%%%%%%%
	
	%%%%%%%%%%% PART 2 %%%%%%%%%%%%%%%%%
	\section{Implement Frontier-Based Exploration System}
	
	
	%%%%%%%%%%%%%%%%%%%%%%%%%%%%%%%%%%%%%%
	
	%%%%%%%%%%% PART 3 %%%%%%%%%%%%%%%%%
	\section{Implement and Investigate Properties of Path Planning Algorithms}
	
	
	%%%%%%%%%%%%%%%%%%%%%%%%%%%%%%%%%%%%%%
	
	%%%%%%%%%%% PART 4 %%%%%%%%%%%%%%%%%
	\section{Information-Theoretic Path Planning}
	
	
	%%%%%%%%%%%%%%%%%%%%%%%%%%%%%%%%%%%%%%
	
	\newpage
	
	\appendix
	\appendixpage
	\addappheadtotoc
	
\end{document}